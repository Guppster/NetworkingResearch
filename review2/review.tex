\documentclass[10pt,]{article}
\usepackage{lmodern}
\usepackage{amssymb,amsmath}
\usepackage{ifxetex,ifluatex}
\usepackage{fixltx2e} % provides \textsubscript
\ifnum 0\ifxetex 1\fi\ifluatex 1\fi=0 % if pdftex
  \usepackage[T1]{fontenc}
  \usepackage[utf8]{inputenc}
\else % if luatex or xelatex
  \ifxetex
    \usepackage{mathspec}
  \else
    \usepackage{fontspec}
  \fi
  \defaultfontfeatures{Ligatures=TeX,Scale=MatchLowercase}
\fi
% use upquote if available, for straight quotes in verbatim environments
\IfFileExists{upquote.sty}{\usepackage{upquote}}{}
% use microtype if available
\IfFileExists{microtype.sty}{%
\usepackage{microtype}
\UseMicrotypeSet[protrusion]{basicmath} % disable protrusion for tt fonts
}{}
\usepackage[margin=1in]{geometry}
\usepackage[unicode=true]{hyperref}
\PassOptionsToPackage{usenames,dvipsnames}{color} % color is loaded by hyperref
\hypersetup{
            pdftitle={Research Review},
            colorlinks=true,
            linkcolor=Maroon,
            citecolor=blue,
            urlcolor=blue,
            breaklinks=true}
\urlstyle{same}  % don't use monospace font for urls
\usepackage{natbib}
\bibliographystyle{plainnat}
\setlength{\emergencystretch}{3em}  % prevent overfull lines
\providecommand{\tightlist}{%
  \setlength{\itemsep}{0pt}\setlength{\parskip}{0pt}}
\setcounter{secnumdepth}{0}
% Redefines (sub)paragraphs to behave more like sections
\ifx\paragraph\undefined\else
\let\oldparagraph\paragraph
\renewcommand{\paragraph}[1]{\oldparagraph{#1}\mbox{}}
\fi
\ifx\subparagraph\undefined\else
\let\oldsubparagraph\subparagraph
\renewcommand{\subparagraph}[1]{\oldsubparagraph{#1}\mbox{}}
\fi

% set default figure placement to htbp
\makeatletter
\def\fps@figure{htbp}
\makeatother



% Stuff I added.
% --------------

\usepackage{indentfirst}
\usepackage[doublespacing]{setspace}
\usepackage{fancyhdr}
\pagestyle{fancy}
\usepackage{layout}   
\lhead{\sc Research Review}
\chead{}
\rhead{\thepage}
\lfoot{}
\cfoot{}
\rfoot{}

\renewcommand{\headrulewidth}{0.0pt}
\renewcommand{\footrulewidth}{0.0pt}

\usepackage{sectsty}
\sectionfont{\centering}
\subsectionfont{\centering}

\newtheorem{hypothesis}{Hypothesis}

% Begin document
% --------------

\begin{document}

\doublespacing

\begin{titlepage}
    \begin{center}
    \line(1,0){300} \\ 
    [0.25in]
    \huge{\bfseries Research Review} \\
    [2mm]
    \line(1,0){200} \\
    [1.5cm] 
    \textsc{\Large Network Operation Cost Model to Achieve Efficient Operation and
Improving Cost Competitiveness} \\
    [0.75cm]
    \textsc{\Large CS4457 Networks II} \\
    [9cm]
    \end{center}
    
    \begin{flushright}
    \textsc{\Large{Gurpreet Singh \\}\normalsize\emph{\# 250674134 \\}\normalsize\emph{Prof.~Haque \\} }
    
    \end{flushright}
    
\end{titlepage}

\newpage

\section{Summary}\label{summary}

The authors of ``Network Operation Cost Model to Achieve Efficient
Operation and Improving Cost Competitiveness'' intended to communicate
the lack of development in cost estimation methodology in
telecommunication operations and improve it with their new
Activity-Based Costing (ABC) method. The paper evaluates as-is network
cost models and derives an improvement plan with a new cost estimation
method.

Before introducing their method, they begin by explaining some of the
costing characteristics used in the telecommunications industry. The
characteristics were: high facility cost, more overhead expense than
direct expense, cost estimation is limited, and government regulations.
Some cost models that are explained in more detail are: Fully
Distributed Costs (FDC), Long Run Incremental Costing (LRIC), Glide
Path, Element Based Model, Activity-Based Costing (ABC) and Building
Block Costing (BBC). With this information the authors provided an
adequate amount of background detail for me to understand the topic and
problem they are focused on solving.

In the next section of the article the authors begin explaining their
proposed ``methodolgy''. \citet{thearticle} It is a slow start to their
explanation as there is a lot of repetition of background information
here. The 4 principles that their methodology is based on are:
Efficiency, Forward Looking, Fairness, and Actual Time-Based. Analyzing
the descriptions of these points I can derive the essence of the
methodology. The proposed changes result in setting a standard cost to
each item, instead of aggregating historical information, and applying
activity-based costing to the model. At this point in the article they
have not provided any convincing data or statistics to prove the
direction they are going is working. Therefore, I am not believing that
ignoring historical data is beneficial.

The figures that are provided on page 1110 are undescriptive and provide
no real value to the article. One of the tables is cut in half and the
article does not adequately describe what it is showing. There is a
figure showing the different stages of their model, which are: define
activity, measure, estimate cost, analyze cost. In this figure it is
visible how vague their process is.

\section{Technical Analysis}\label{technical-analysis}

In the cost elimination steps outlined in part B of section 3 its seen
that the authors are attempting to use feedback data from the system to
determine what the best figures are. This process feels like using ABC
but eliminating all previous historical data and starting to gather new
data from scratch. This cannot be good for large telecommunication
companies because they would lose too much money at the start of the
cost elimination process to make it worth the switch.

The authors define a very vague process that can be interpreted in many
different ways. In essence the process is to define a unit activity that
costs money in a company, measure how many resources it consumes, make a
cost estimate for it and then after completing the unit activity, update
the existing model with the real cost. The goal with this model is to
keep increasing the amount of data available to the model so it can make
more informed decisions about the company's costs. This seems like an
achievable goal. If the article provided more concrete details on how it
can produce this better than ABC already does, it would have been more
convincing.

There are some comparisons to the existing ABC model after the model
description, explaining how their model has more data and is more
accurate in showing the largest costs in a company. This is interesting
but the main purpose of the paper is to reduce the costing, and although
these are related points, they don't directly link it to a large
reduction in costs. This sounds too similar to ABC for the authors to
even have a basis for research.

From reading their technical details, I do not believe this model would
help telecommunication companies improve cost efficiency because of the
upfront costs associated with putting the model in place.

\section{Suggestions and
Presentation}\label{suggestions-and-presentation}

The presentation of this article was very poor. The general readability
of the article greatly suffered due to lack of proofreading which was
seen throughout the article. There were many grammatical errors that
made the article almost unreadable. Sentences had poor cohesion and it
was not easy to determine the topic of each paragraph. On top of the
poor English that was observed in the article, there was also a lack of
figures describing the model they are proposing. Due to these qualities,
the true meaning of the article is hard to understand. The validity of
the paper suffers because of the many different ways each paragraph
could be interpreted.

The authors could have greatly improved this article by getting it
proofread and reorganizing their thoughts. From an article that is
proposing a new way of approaching an old costing method, I would have
expected direct comparisons to ABC and quantitative proof of how they
can improve upon it.

\bibliography{bibliography.bib}

\end{document}

