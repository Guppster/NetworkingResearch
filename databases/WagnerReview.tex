\documentclass[11pt,]{article}
\usepackage{lmodern}
\usepackage{amssymb,amsmath}
\usepackage{ifxetex,ifluatex}
\usepackage{fixltx2e} % provides \textsubscript
\ifnum 0\ifxetex 1\fi\ifluatex 1\fi=0 % if pdftex
  \usepackage[T1]{fontenc}
  \usepackage[utf8]{inputenc}
\else % if luatex or xelatex
  \ifxetex
    \usepackage{mathspec}
  \else
    \usepackage{fontspec}
  \fi
  \defaultfontfeatures{Ligatures=TeX,Scale=MatchLowercase}
\fi
% use upquote if available, for straight quotes in verbatim environments
\IfFileExists{upquote.sty}{\usepackage{upquote}}{}
% use microtype if available
\IfFileExists{microtype.sty}{%
\usepackage{microtype}
\UseMicrotypeSet[protrusion]{basicmath} % disable protrusion for tt fonts
}{}
\usepackage[margin=1in]{geometry}
\usepackage[unicode=true]{hyperref}
\PassOptionsToPackage{usenames,dvipsnames}{color} % color is loaded by hyperref
\hypersetup{
            pdftitle={Paper Review},
            colorlinks=true,
            linkcolor=Maroon,
            citecolor=blue,
            urlcolor=blue,
            breaklinks=true}
\urlstyle{same}  % don't use monospace font for urls
\usepackage{natbib}
\bibliographystyle{plainnat}
\setlength{\emergencystretch}{3em}  % prevent overfull lines
\providecommand{\tightlist}{%
  \setlength{\itemsep}{0pt}\setlength{\parskip}{0pt}}
\setcounter{secnumdepth}{0}
% Redefines (sub)paragraphs to behave more like sections
\ifx\paragraph\undefined\else
\let\oldparagraph\paragraph
\renewcommand{\paragraph}[1]{\oldparagraph{#1}\mbox{}}
\fi
\ifx\subparagraph\undefined\else
\let\oldsubparagraph\subparagraph
\renewcommand{\subparagraph}[1]{\oldsubparagraph{#1}\mbox{}}
\fi

% set default figure placement to htbp
\makeatletter
\def\fps@figure{htbp}
\makeatother



% Stuff I added.
% --------------

\usepackage{indentfirst}
\usepackage[doublespacing]{setspace}
\usepackage{fancyhdr}
\pagestyle{fancy}
\usepackage{layout}   
\lhead{\sc Paper Review}
\chead{}
\rhead{\thepage}
\lfoot{}
\cfoot{}
\rfoot{}

\renewcommand{\headrulewidth}{0.0pt}
\renewcommand{\footrulewidth}{0.0pt}

\usepackage{sectsty}
\sectionfont{\centering}
\subsectionfont{\centering}

\newtheorem{hypothesis}{Hypothesis}

% Begin document
% --------------

\begin{document}

\doublespacing

\begin{titlepage}
    \begin{center}
    \line(1,0){300} \\ 
    [0.25in]
    \huge{\bfseries Paper Review} \\
    [2mm]
    \line(1,0){200} \\
    [1.5cm] 
    \textsc{\Large Database Forensic Analysis with DBCarver} \\
    [0.75cm]
    \textsc{\Large } \\
    [9cm]
    \end{center}
    
    \begin{flushright}
    \textsc{\Large{Gurpreet Singh \\}\normalsize\emph{\ February 11, 2018 \\}\normalsize\emph{CS4414 \\} }
    
    \end{flushright}
    
\end{titlepage}

\newpage

\hypertarget{jot-notes}{%
\subsection{Jot notes}\label{jot-notes}}

\hypertarget{abstract}{%
\subsubsection{Abstract}\label{abstract}}

\begin{itemize}
\tightlist
\item
  Forensics needs to be effective on databases that are configured
  incorrectly and have no protection techniques integrated in them.
  Sometimes they don't even have logging
\item
  Paper talks about a tool called DBCarver.
\item
  Reconstructs database context from a database image without using any
  log or system data.
\item
  Uses page carving to reconstruct both query-able data and deleted
  data.
\item
  Allows investigators to conduct new types of analysis on the data
\end{itemize}

\hypertarget{introduction}{%
\subsubsection{Introduction}\label{introduction}}

\begin{itemize}
\item
  Since most large applications use databases, large scale cyber crime
  almost always involves databases.
\item
  A related concept is file carving, which retrieves files from storage
  on a file system even if they are deleted or corrupted
\item
  Relational databases store data in pages
\item
  We can reconstruct pages to retrieve data
\item
  Targets extracting data maintained by the running database instead of
  recovering original
\item
  Paper considers two scenarios, database is good or bad
\item
\end{itemize}

\hypertarget{summary}{%
\section{Summary}\label{summary}}

\hypertarget{strengths-and-weaknesses}{%
\section{Strengths and Weaknesses}\label{strengths-and-weaknesses}}

\end{document}

