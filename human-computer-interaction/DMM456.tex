\documentclass[11pt,]{article}
\usepackage{lmodern}
\usepackage{amssymb,amsmath}
\usepackage{ifxetex,ifluatex}
\usepackage{fixltx2e} % provides \textsubscript
\ifnum 0\ifxetex 1\fi\ifluatex 1\fi=0 % if pdftex
  \usepackage[T1]{fontenc}
  \usepackage[utf8]{inputenc}
\else % if luatex or xelatex
  \ifxetex
    \usepackage{mathspec}
  \else
    \usepackage{fontspec}
  \fi
  \defaultfontfeatures{Ligatures=TeX,Scale=MatchLowercase}
\fi
% use upquote if available, for straight quotes in verbatim environments
\IfFileExists{upquote.sty}{\usepackage{upquote}}{}
% use microtype if available
\IfFileExists{microtype.sty}{%
\usepackage{microtype}
\UseMicrotypeSet[protrusion]{basicmath} % disable protrusion for tt fonts
}{}
\usepackage[margin=1in]{geometry}
\usepackage[unicode=true]{hyperref}
\PassOptionsToPackage{usenames,dvipsnames}{color} % color is loaded by hyperref
\hypersetup{
            pdftitle={Designing with the Mind in Mind},
            colorlinks=true,
            linkcolor=Maroon,
            citecolor=blue,
            urlcolor=blue,
            breaklinks=true}
\urlstyle{same}  % don't use monospace font for urls
\usepackage{natbib}
\bibliographystyle{plainnat}
\setlength{\emergencystretch}{3em}  % prevent overfull lines
\providecommand{\tightlist}{%
  \setlength{\itemsep}{0pt}\setlength{\parskip}{0pt}}
\setcounter{secnumdepth}{0}
% Redefines (sub)paragraphs to behave more like sections
\ifx\paragraph\undefined\else
\let\oldparagraph\paragraph
\renewcommand{\paragraph}[1]{\oldparagraph{#1}\mbox{}}
\fi
\ifx\subparagraph\undefined\else
\let\oldsubparagraph\subparagraph
\renewcommand{\subparagraph}[1]{\oldsubparagraph{#1}\mbox{}}
\fi

% set default figure placement to htbp
\makeatletter
\def\fps@figure{htbp}
\makeatother



% Stuff I added.
% --------------

\usepackage{indentfirst}
\usepackage{fancyhdr}
\pagestyle{fancy}
\usepackage{layout}   
\lhead{\sc Reading Summary}
\chead{}
\rhead{\thepage}
\lfoot{}
\cfoot{}
\rfoot{}

\renewcommand{\headrulewidth}{0.0pt}
\renewcommand{\footrulewidth}{0.0pt}

\usepackage{sectsty}
\sectionfont{\centering}
\subsectionfont{\centering}

\newtheorem{hypothesis}{Hypothesis}

% Begin document
% --------------

\begin{document}


\begin{titlepage}
    \begin{center}
    \line(1,0){300} \\ 
    [0.25in]
    \huge{\bfseries Designing with the Mind in Mind} \\
    [2mm]
    \line(1,0){200} \\
    [1.5cm] 
    \textsc{\Large Chapters 4, 5 and 6} \\
    [0.75cm]
    \textsc{\Large Reading Summary} \\
    [12cm]
    \end{center}
    
    \begin{flushright}
    \textsc{\Large{Gurpreet Singh \\}\normalsize\emph{\ January 28, 2018 \\}\normalsize\emph{CS4474 \\} }
    
    \end{flushright}

\end{titlepage}

\newpage
\shipout\null

\hypertarget{chapter-4}{%
\section{Chapter 4}\label{chapter-4}}

\hypertarget{reading-is-unnatural}{%
\subsection{Reading is Unnatural}\label{reading-is-unnatural}}

\hypertarget{wired-for-language-not-reading}{%
\subsubsection{Wired for Language not
Reading}\label{wired-for-language-not-reading}}

We are born to speak languages and learn to speak quickly in our
childhood but when we get older it is as difficult as any other skill A
different brain area is used for this type of learning after childhood.
Most people learn how to speak but some people never learn how to read.
People may not be good at reading because of \{basically my problems
irl\}

Reading is teaching our brains how to recognize patterns. Morphemes are
combinations of letters we recognize and put together to make words

\hypertarget{feature-driven-vs-context-driven}{%
\subsubsection{Feature driven vs context
driven}\label{feature-driven-vs-context-driven}}

Feature driven is bottom up and sometimes called ``context free''. It
starts by identifying simple features like lines and uses them to make
complex objects like words. This type of reading is faster than context
driven despite instinct.

Context driven is top down and opposite of feature driven. You recognize
a morpheme, word or sentence and treat it like a single object to get
it's meaning. An example of this is something you would read often.
Mainly a backup method for reading.

\hypertarget{skilled-vs-unskilled-reading}{%
\subsubsection{Skilled vs Unskilled
reading}\label{skilled-vs-unskilled-reading}}

``Skilled readers'' use feature driven mostly and fallback to context
driven ``Unskilled readers'' use context driven because feature driven
is slow for them.

Not enough experience reading as a child causes unskilled reading

\hypertarget{poor-information-design-disprupts-reading}{%
\subsubsection{Poor information design disprupts
reading}\label{poor-information-design-disprupts-reading}}

Poor presentation can block reading for unskilled readers. The following
are things that disrupt reading:

\begin{itemize}
\tightlist
\item
  Uncommon words
\item
  Difficult scripts and typefaces
\item
  Tiny Fonts
\item
  Text on noisy background
\item
  Visual noise from too much text
\item
  Centered text is hard for eyes to follow
\end{itemize}

Any combination of the above can make for a very unreadable design
Support both reading styles by avoiding the problems listed above.

\hypertarget{most-reading-is-unnecessary}{%
\subsubsection{Most reading is
unnecessary}\label{most-reading-is-unnecessary}}

Most software has too much text in the instructions and people don't
read it anyway. You can be just as clear with less text. Provide a brief
overview and let users request more detail if they want.

\hypertarget{chapter-5}{%
\section{Chapter 5}\label{chapter-5}}

\hypertarget{our-color-vision-is-limited}{%
\subsection{Our Color Vision is
Limited}\label{our-color-vision-is-limited}}

\hypertarget{optimized-for-edge-contrast}{%
\subsubsection{Optimized for Edge
Contrast}\label{optimized-for-edge-contrast}}

We can see something better if it is high contrast and brightness isn't
as important

\hypertarget{discrimination-based-on-presentation}{%
\subsubsection{Discrimination based on
Presentation}\label{discrimination-based-on-presentation}}

The paler (less saturated) two colors are, the harder it is to tell them
apart

The smaller and thinner objects are, the harder it is to distinguish
their colors

The more separated color patches are, the more difficult it is to
distinguish their colors

\hypertarget{color-blindness}{%
\subsubsection{Color-Blindness}\label{color-blindness}}

Converting an image to grayscale can help you design applications for
color blind people

\hypertarget{external-factors}{%
\subsubsection{External Factors}\label{external-factors}}

Each color display can show something different just because of the
technology. Grayscale displays can also have this problem. Display angle
can make different images look different. TN panels lol. Ambient
illumination can wash out colors, enough to make color displays into
grayscale

\hypertarget{guidelines-for-using-color}{%
\subsubsection{Guidelines for using
Color}\label{guidelines-for-using-color}}

\begin{itemize}
\tightlist
\item
  Distinguish colors by saturation and brightness as well as hue, avoid
  subtle color differences
\item
  Use distinctive colors, increase contrast
\item
  Avoid color pairs that color-blind people cannot distinguish
\item
  Use color redundantly with other hues
\item
  Separate string opponent colors
\end{itemize}

\hypertarget{chapter-6}{%
\section{Chapter 6}\label{chapter-6}}

\hypertarget{our-peripheral-vision-is-poor}{%
\subsection{Our Peripheral Vision is
Poor}\label{our-peripheral-vision-is-poor}}

\hypertarget{resolution-of-the-fovea-vs-periphery}{%
\subsubsection{Resolution of the Fovea vs
Periphery}\label{resolution-of-the-fovea-vs-periphery}}

There is a lower resolution on the outside of your eye than on the
inside of your eye. You see great in the center and not so great on the
outsides. We can discriminate colors better in the center as well.

\hypertarget{good-for-anything}{%
\subsubsection{Good for Anything?}\label{good-for-anything}}

It provides low resolution cues to help guide our eyes. It sees fuzzy
things and then we focus on that fuzzy thing if it looks like what we
want. Its good at detecting motion

\hypertarget{computer-user-interfaces}{%
\subsubsection{Computer User
Interfaces}\label{computer-user-interfaces}}

Errors are often missed if they aren't placed in the center or where the
user is looking. Make sure error messages are popping up where the user
is looking like the ``Login'' button.

\hypertarget{techniques-to-make-messages-visible}{%
\subsubsection{Techniques to make messages
visible}\label{techniques-to-make-messages-visible}}

\begin{itemize}
\tightlist
\item
  Put it where the users are looking (like i said above)
\item
  Mark the error with what it is referencing
\item
  Use an error symbol because people recognize it
\item
  Reserve red for errors because people recognize it
\end{itemize}

\hypertarget{more-dangerous-ways-to-get-users-attention}{%
\paragraph{More dangerous ways to get user's
attention}\label{more-dangerous-ways-to-get-users-attention}}

\begin{itemize}
\tightlist
\item
  Pop-up message
\end{itemize}

This makes it hard to miss because it steals the focus from the
application window so the user HAS to see it.

\begin{itemize}
\tightlist
\item
  Use Sound
\item
  Flash or Wiggle
\end{itemize}

These two will alert the user better because they are such obvious
indicators of action. They will cause the user to scan for the error
These should be reserved for very critical errors as they can become
very annoying

\end{document}

