\documentclass[11pt,]{article}
\usepackage{lmodern}
\usepackage{amssymb,amsmath}
\usepackage{ifxetex,ifluatex}
\usepackage{fixltx2e} % provides \textsubscript
\ifnum 0\ifxetex 1\fi\ifluatex 1\fi=0 % if pdftex
  \usepackage[T1]{fontenc}
  \usepackage[utf8]{inputenc}
\else % if luatex or xelatex
  \ifxetex
    \usepackage{mathspec}
  \else
    \usepackage{fontspec}
  \fi
  \defaultfontfeatures{Ligatures=TeX,Scale=MatchLowercase}
\fi
% use upquote if available, for straight quotes in verbatim environments
\IfFileExists{upquote.sty}{\usepackage{upquote}}{}
% use microtype if available
\IfFileExists{microtype.sty}{%
\usepackage{microtype}
\UseMicrotypeSet[protrusion]{basicmath} % disable protrusion for tt fonts
}{}
\usepackage[margin=1in]{geometry}
\usepackage[unicode=true]{hyperref}
\PassOptionsToPackage{usenames,dvipsnames}{color} % color is loaded by hyperref
\hypersetup{
            pdftitle={Designing with the Mind in Mind},
            colorlinks=true,
            linkcolor=Maroon,
            citecolor=blue,
            urlcolor=blue,
            breaklinks=true}
\urlstyle{same}  % don't use monospace font for urls
\usepackage{natbib}
\bibliographystyle{plainnat}
\setlength{\emergencystretch}{3em}  % prevent overfull lines
\providecommand{\tightlist}{%
  \setlength{\itemsep}{0pt}\setlength{\parskip}{0pt}}
\setcounter{secnumdepth}{0}
% Redefines (sub)paragraphs to behave more like sections
\ifx\paragraph\undefined\else
\let\oldparagraph\paragraph
\renewcommand{\paragraph}[1]{\oldparagraph{#1}\mbox{}}
\fi
\ifx\subparagraph\undefined\else
\let\oldsubparagraph\subparagraph
\renewcommand{\subparagraph}[1]{\oldsubparagraph{#1}\mbox{}}
\fi

% set default figure placement to htbp
\makeatletter
\def\fps@figure{htbp}
\makeatother



% Stuff I added.
% --------------

\usepackage{indentfirst}
\usepackage{fancyhdr}
\pagestyle{fancy}
\usepackage{layout}   
\lhead{\sc Reading Summary}
\chead{}
\rhead{\thepage}
\lfoot{}
\cfoot{}
\rfoot{}

\renewcommand{\headrulewidth}{0.0pt}
\renewcommand{\footrulewidth}{0.0pt}

\usepackage{sectsty}
\sectionfont{\centering}
\subsectionfont{\centering}

\newtheorem{hypothesis}{Hypothesis}

% Begin document
% --------------

\begin{document}


\begin{titlepage}
    \begin{center}
    \line(1,0){300} \\ 
    [0.25in]
    \huge{\bfseries Designing with the Mind in Mind} \\
    [2mm]
    \line(1,0){200} \\
    [1.5cm] 
    \textsc{\Large Chapters 13, 14, 15 and Appendix} \\
    [0.75cm]
    \textsc{\Large Reading Summary} \\
    [12cm]
    \end{center}
    
    \begin{flushright}
    \textsc{\Large{Gurpreet Singh \\}\normalsize\emph{\ February 10, 2018 \\}\normalsize\emph{CS4474 \\} }
    
    \end{flushright}

\end{titlepage}

\newpage
\shipout\null

\hypertarget{chapter-13}{%
\section{Chapter 13}\label{chapter-13}}

\hypertarget{our-hand-eye-coordination-follows-laws}{%
\subsection{Our Hand-Eye Coordination Follows
Laws}\label{our-hand-eye-coordination-follows-laws}}

\hypertarget{fitts-law-pointing-at-displayed-targets}{%
\subsubsection{Fitts Law: Pointing at Displayed
Targets}\label{fitts-law-pointing-at-displayed-targets}}

The larger your target on the screen and the closer it is to your
starting point, the faster you can reach it. The formula for this is T =
a + b log(1+ D/W) where T is the time required to point to something, D
is the distance, W is the width of the target. Applies to all pointing
devices and all people. a and b are variables accounting for differences
between people and pointing devices.

Design implications

\begin{itemize}
\tightlist
\item
  Use decently sized click targets
\item
  Make the click-boxes big enough to not frustrate the user
\item
  Accept clicks on labels and graphics to improve hit rate
\item
  Use margins around click targets to avoid misclicks
\item
  Placing targets near the edge of the screen makes them easier to click
\end{itemize}

\hypertarget{steering-law-moving-pointers-along-constrained-paths}{%
\subsubsection{Steering Law: Moving Pointers Along Constrained
Paths}\label{steering-law-moving-pointers-along-constrained-paths}}

Making users move a pointer along a narrow path is slower than allowing
them a larger path to move the pointer. Design implications include
making drop down menu entries wider and higher, avoiding pull-right
menus and page rulers.

This law mainly describes the absolute pain it is to use pull-right
menus. That feeling when you are 4 clicks deep into a pull-right and you
go off UI and have to restart.

Scroll bars and page rulers are also another area where this problem is
present.

\hypertarget{chapter-14}{%
\section{Chapter 14}\label{chapter-14}}

\hypertarget{we-have-time-requirements}{%
\subsection{We have Time Requirements}\label{we-have-time-requirements}}

\hypertarget{responsivness-defined}{%
\subsubsection{Responsivness defined}\label{responsivness-defined}}

Responsiveness is how quickly a system responds to the user's
interactions. A system can be responsive even with poor performance. You
should use callbacks to let the UI continue operating and let the user
know that something will finish. When a system isnt responsive it cant
meet the time deadlines of the human.

\hypertarget{time-constraints-for-the-human-brain}{%
\subsubsection{Time constraints for the human
brain}\label{time-constraints-for-the-human-brain}}

1 milisecond is the shortest amount of silence you can detect. 5
miliseconds is the shortest amount of time you can see a visual change
to be effected by it 80 miliseconds is how long it takes you to flinch
to something. Then it takes 100 miliseconds to fully process something
you have seen. Your system has to react to an action within 140
miliseconds (1.4 seconds) for the user to understand that their action
had an effect on your system. Your attention takes 500 miliseconds to
reset from one item to the next. It takes 700 miliseconds to do a motor
action after observing something visual. Any gap in conversation longer
than 1 second is awkward. A sub-mental task can take up to a maximum of
10 seconds.

Interactive systems need to keep the above constraints in mind when
engineering.

\hypertarget{hci-implications}{%
\subsubsection{HCI implications}\label{hci-implications}}

The guidelines for this chapter are:

\begin{itemize}
\tightlist
\item
  React to user interactions instantly
\item
  Indicate background processing
\item
  Use callbacks and dont hold UI in focus when processing
\item
  Animate smoothly
\item
  Allow users to cancel processing
\item
  provide a ETA for processing
\end{itemize}

Additional implications include - use busy indicators (spinning circles
and loading bars) - Know when it is allowed to use a delay and when it
will bother the user (sometimes its acceptable if they are doing
something hard) - Display important data first and avoid delays by
prompting for extra information instead of preprocessing everything for
viewing - When the user is not doing anything directly, you can process
information in the background and be ready for the next actions the user
may take - Monitor how long your application has taken and if it falls
into an acceptable time duration

\hypertarget{appendix}{%
\section{Appendix}\label{appendix}}

\hypertarget{well-known-user-interface-design-rules}{%
\section{Well-Known User Interface Design
Rules}\label{well-known-user-interface-design-rules}}

The following are the major tips compiled from all appendix items
without duplicates

\begin{itemize}
\tightlist
\item
  Always provide the user feedback in a way that they cannot be confused
  about the systems state
\item
  Different types of interactions should use different styles of menus
  so they can be uniquely identified by the user in memory
\item
  Always provide the ability to undo actions
\item
  System should do stuff in a consistent way so the user has to remember
  less of the system
\item
  Make the users feel like they are in control at all times.
\item
  Help users recover from errors instead of quitting and making them
  restart
\item
  Provide lots of documentation for exploration
\item
  Function first, Presentation later
\item
  Design for common case not the complex case
\item
  Try stuff out on users, then fix it
\end{itemize}

\end{document}

