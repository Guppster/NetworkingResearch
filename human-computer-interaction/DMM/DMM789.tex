\documentclass[11pt,]{article}
\usepackage{lmodern}
\usepackage{amssymb,amsmath}
\usepackage{ifxetex,ifluatex}
\usepackage{fixltx2e} % provides \textsubscript
\ifnum 0\ifxetex 1\fi\ifluatex 1\fi=0 % if pdftex
  \usepackage[T1]{fontenc}
  \usepackage[utf8]{inputenc}
\else % if luatex or xelatex
  \ifxetex
    \usepackage{mathspec}
  \else
    \usepackage{fontspec}
  \fi
  \defaultfontfeatures{Ligatures=TeX,Scale=MatchLowercase}
\fi
% use upquote if available, for straight quotes in verbatim environments
\IfFileExists{upquote.sty}{\usepackage{upquote}}{}
% use microtype if available
\IfFileExists{microtype.sty}{%
\usepackage{microtype}
\UseMicrotypeSet[protrusion]{basicmath} % disable protrusion for tt fonts
}{}
\usepackage[margin=1in]{geometry}
\usepackage[unicode=true]{hyperref}
\PassOptionsToPackage{usenames,dvipsnames}{color} % color is loaded by hyperref
\hypersetup{
            pdftitle={Designing with the Mind in Mind},
            colorlinks=true,
            linkcolor=Maroon,
            citecolor=blue,
            urlcolor=blue,
            breaklinks=true}
\urlstyle{same}  % don't use monospace font for urls
\usepackage{natbib}
\bibliographystyle{plainnat}
\setlength{\emergencystretch}{3em}  % prevent overfull lines
\providecommand{\tightlist}{%
  \setlength{\itemsep}{0pt}\setlength{\parskip}{0pt}}
\setcounter{secnumdepth}{0}
% Redefines (sub)paragraphs to behave more like sections
\ifx\paragraph\undefined\else
\let\oldparagraph\paragraph
\renewcommand{\paragraph}[1]{\oldparagraph{#1}\mbox{}}
\fi
\ifx\subparagraph\undefined\else
\let\oldsubparagraph\subparagraph
\renewcommand{\subparagraph}[1]{\oldsubparagraph{#1}\mbox{}}
\fi

% set default figure placement to htbp
\makeatletter
\def\fps@figure{htbp}
\makeatother



% Stuff I added.
% --------------

\usepackage{indentfirst}
\usepackage{fancyhdr}
\pagestyle{fancy}
\usepackage{layout}   
\lhead{\sc Reading Summary}
\chead{}
\rhead{\thepage}
\lfoot{}
\cfoot{}
\rfoot{}

\renewcommand{\headrulewidth}{0.0pt}
\renewcommand{\footrulewidth}{0.0pt}

\usepackage{sectsty}
\sectionfont{\centering}
\subsectionfont{\centering}

\newtheorem{hypothesis}{Hypothesis}

% Begin document
% --------------

\begin{document}


\begin{titlepage}
    \begin{center}
    \line(1,0){300} \\ 
    [0.25in]
    \huge{\bfseries Designing with the Mind in Mind} \\
    [2mm]
    \line(1,0){200} \\
    [1.5cm] 
    \textsc{\Large Chapters 7, 8 and 9} \\
    [0.75cm]
    \textsc{\Large Reading Summary} \\
    [12cm]
    \end{center}
    
    \begin{flushright}
    \textsc{\Large{Gurpreet Singh \\}\normalsize\emph{\ January 29, 2018 \\}\normalsize\emph{CS4474 \\} }
    
    \end{flushright}

\end{titlepage}

\newpage
\shipout\null

\hypertarget{chapter-7}{%
\section{Chapter 7}\label{chapter-7}}

\hypertarget{our-attention-is-limited-our-memory-is-imperfect}{%
\subsection{Our Attention is Limited Our Memory is
Imperfect}\label{our-attention-is-limited-our-memory-is-imperfect}}

\hypertarget{short-vs-long-term}{%
\subsubsection{Short vs Long term}\label{short-vs-long-term}}

Short term ranges from a faction of a second to several seconds up to a
minute. Long term memory covers the rest of the memory situations. They
are not seperate physical stores of memory in the brain.

\hypertarget{long-term}{%
\paragraph{Long term}\label{long-term}}

How your brain reacts to senses greatly depends on the context in which
you are and the features you are experiencing. The more similar two
situations, the more overlap there is between brain activity. The more a
neural memory pattern is reactivated the stronger it becomes and
therefore is easier to reactivate and this signifies long term memory.
Neural patterns with high recall, strong, and easy to reactivate.

\hypertarget{error-prone}{%
\paragraph{Error prone}\label{error-prone}}

Human brain has unlimited memory. Everything stored in heavily
compressed in lossy format. Different memories have different levels of
detail

\hypertarget{weighted-by-emotions}{%
\paragraph{Weighted by emotions}\label{weighted-by-emotions}}

Emotions can change how well you remember something and what you feel
when you think of that memory

\hypertarget{retroactively-alterable}{%
\paragraph{Retroactively alterable}\label{retroactively-alterable}}

Your memory of an event can change over time and be morphed into other
memories. Nothing is 100 percent accurate.

\hypertarget{implications-of-long-term-memory}{%
\subsubsection{Implications of long term
memory}\label{implications-of-long-term-memory}}

Avoid making software that burdens longterm memory by not making people
remember things often. Password authentication breaks this rule.
Security questions also break this rule. Editor shortcut key
combinations break this rule. Keep your shortcuts consistent.

\hypertarget{short-term}{%
\paragraph{Short term}\label{short-term}}

Short term memory is not a store and instead is just a temporary place
where senses go to be worked on. Each sense has it's own small short
term memory. The brain fetches data from these sense caches like a
queue. Also called working memory and is equal to the focus of our
attention. Has low capacity and is volatile. Anything one can be aware
of can be stored in short term memory Can contain 4 plus or minus two
items if they arent chunked together

\hypertarget{implications-of-short-term-memory}{%
\subsubsection{Implications of short term
memory}\label{implications-of-short-term-memory}}

Dont require people to remember too many things at one time

\hypertarget{modes}{%
\paragraph{Modes}\label{modes}}

Avoid modes, or provide mode-feedback because people can't remember.
Modes are like VIM, which is great, advantages obvious. Disadvantages
are you can run into mode-errors. Not knowing which mode you're in.

\hypertarget{search-results}{%
\paragraph{Search Results}\label{search-results}}

Remember to show the search terms in the search results page because
people can't remember what they just searched for a few seconds ago.

\hypertarget{instructions}{%
\paragraph{Instructions}\label{instructions}}

Show instructions to the user when they need to use them because they
cannot remember all the steps when they are on the next page.

\hypertarget{chapter-8}{%
\section{Chapter 8}\label{chapter-8}}

\hypertarget{limits-on-attention-shape-thought-and-action}{%
\subsection{Limits on Attention, Shape, Thought and
Action}\label{limits-on-attention-shape-thought-and-action}}

\hypertarget{we-focus-on-our-goals-not-our-tools}{%
\subsubsection{We focus on our goals, not our
tools}\label{we-focus-on-our-goals-not-our-tools}}

We put all our attention into achiving our goal and very little on how
to use our tools. Therefore we sometimes continue using our tools
inefficiently as long as we achieve our goal.

\hypertarget{we-use-external-aids}{%
\subsubsection{We use external aids}\label{we-use-external-aids}}

We change our environment to store information inside of it instead of
using our short or long term memory. Therefore software should visually
indicate where the user is in their task so they don't have to memorize
anything. Ex. read and unread emails being highlighted and grayed out
respectively

\hypertarget{we-follow-information-scent-toward-our-goal}{%
\subsubsection{We follow information scent toward our
goal}\label{we-follow-information-scent-toward-our-goal}}

Humans click on things that indicate that they will lead to the goal
they want to achieve.

\hypertarget{we-prefer-familiar-paths}{%
\subsubsection{We prefer familiar
paths}\label{we-prefer-familiar-paths}}

People do what they are used to instead of exploring new paths through
the software because their attention is limited. Problem solving places
heavy load on attention and short-term memory. Classic case of vim users
sticking to the few commands they know to achieve full functionality
instead of adding to their list of known commands

\hypertarget{our-thought-cycle-goal-execute-evaluate}{%
\subsubsection{Our thought cycle: Goal, Execute,
Evaluate}\label{our-thought-cycle-goal-execute-evaluate}}

We form a goal. We execute actions to try and make progress toward that
goal. We evaluate whether the actions worked and repeat until goal
reached.

Provide clear paths to the goal. Software concepts should be based on
the task rather than the code. Provide feedback to the user so they can
evaluate correctly

\hypertarget{after-we-achieve-a-task-we-forget-cleanup}{%
\subsubsection{After we achieve a task, we forget
cleanup}\label{after-we-achieve-a-task-we-forget-cleanup}}

Once we are out of the goal, execute, evaluate loop for one specific
goal, we forget about that thought stream and don't clean up any mess we
made while reaching that goal. To avoid this, interactive systems should
help the human clean up or atleast remind them in the evaluate portion
of the loop. If possible the system should clean up for the human so you
can save their attention for more important tasks.

\hypertarget{chapter-9}{%
\section{Chapter 9}\label{chapter-9}}

\hypertarget{recognition-easy-recall-hard}{%
\subsection{Recognition Easy, Recall
hard}\label{recognition-easy-recall-hard}}

\hypertarget{recognition-is-easy}{%
\subsubsection{Recognition is Easy}\label{recognition-is-easy}}

Patterns of activity can be activated in two ways: By more perceptions
coming in from the senses, and by other brain activity. If a new
perception comes in that is similar to an old you, you have recognition.
We can quickly recognize faces.

\hypertarget{recall-is-hard}{%
\subsubsection{Recall is hard}\label{recall-is-hard}}

Recall is trying to access long term memory without reactivating old
neural patterns with a new instance. It is possible but much harder. It
is also prone to more errors. People use mind palaces to help with
recall.

\hypertarget{recognition-vs-recall}{%
\subsubsection{Recognition vs Recall}\label{recognition-vs-recall}}

Seeing something and choosing it is easier than recalling something and
typing it so software should do the recalling for the user because it
has memory. People can recognize pictures easily so use those instead of
text to convey function.

\hypertarget{use-thumbnail-images-to-depict-full-sized-images}{%
\paragraph{Use thumbnail images to depict full-sized
images}\label{use-thumbnail-images-to-depict-full-sized-images}}

Recognition doesn't care what size the objects are, we can still
recognize them. Therefore to fit more data into vision at once,
thumbnails are a good idea.

\hypertarget{the-larger-the-number-of-people-who-will-use-a-function-the-more-visible-the-function-should-be}{%
\paragraph{The larger the number of people who will use a function, the
more visible the function should
be}\label{the-larger-the-number-of-people-who-will-use-a-function-the-more-visible-the-function-should-be}}

Make it so the least amount of people have to recall where a function is
located in software. If it is commonly used you don't want people to
fail because it will add up to a lot of people.

\hypertarget{use-visual-cues-to-let-users-recognize-where-they-are}{%
\paragraph{Use visual cues to let users recognize where they
are}\label{use-visual-cues-to-let-users-recognize-where-they-are}}

Visual recognition is fast just like pictures mentioned above. Slight
changes is visuals can help users find out which part of your
application they are currently using.

\end{document}

