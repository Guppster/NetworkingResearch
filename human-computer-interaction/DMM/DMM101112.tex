\documentclass[11pt,]{article}
\usepackage{lmodern}
\usepackage{amssymb,amsmath}
\usepackage{ifxetex,ifluatex}
\usepackage{fixltx2e} % provides \textsubscript
\ifnum 0\ifxetex 1\fi\ifluatex 1\fi=0 % if pdftex
  \usepackage[T1]{fontenc}
  \usepackage[utf8]{inputenc}
\else % if luatex or xelatex
  \ifxetex
    \usepackage{mathspec}
  \else
    \usepackage{fontspec}
  \fi
  \defaultfontfeatures{Ligatures=TeX,Scale=MatchLowercase}
\fi
% use upquote if available, for straight quotes in verbatim environments
\IfFileExists{upquote.sty}{\usepackage{upquote}}{}
% use microtype if available
\IfFileExists{microtype.sty}{%
\usepackage{microtype}
\UseMicrotypeSet[protrusion]{basicmath} % disable protrusion for tt fonts
}{}
\usepackage[margin=1in]{geometry}
\usepackage[unicode=true]{hyperref}
\PassOptionsToPackage{usenames,dvipsnames}{color} % color is loaded by hyperref
\hypersetup{
            pdftitle={Designing with the Mind in Mind},
            colorlinks=true,
            linkcolor=Maroon,
            citecolor=blue,
            urlcolor=blue,
            breaklinks=true}
\urlstyle{same}  % don't use monospace font for urls
\usepackage{natbib}
\bibliographystyle{plainnat}
\setlength{\emergencystretch}{3em}  % prevent overfull lines
\providecommand{\tightlist}{%
  \setlength{\itemsep}{0pt}\setlength{\parskip}{0pt}}
\setcounter{secnumdepth}{0}
% Redefines (sub)paragraphs to behave more like sections
\ifx\paragraph\undefined\else
\let\oldparagraph\paragraph
\renewcommand{\paragraph}[1]{\oldparagraph{#1}\mbox{}}
\fi
\ifx\subparagraph\undefined\else
\let\oldsubparagraph\subparagraph
\renewcommand{\subparagraph}[1]{\oldsubparagraph{#1}\mbox{}}
\fi

% set default figure placement to htbp
\makeatletter
\def\fps@figure{htbp}
\makeatother



% Stuff I added.
% --------------

\usepackage{indentfirst}
\usepackage{fancyhdr}
\pagestyle{fancy}
\usepackage{layout}   
\lhead{\sc Reading Summary}
\chead{}
\rhead{\thepage}
\lfoot{}
\cfoot{}
\rfoot{}

\renewcommand{\headrulewidth}{0.0pt}
\renewcommand{\footrulewidth}{0.0pt}

\usepackage{sectsty}
\sectionfont{\centering}
\subsectionfont{\centering}

\newtheorem{hypothesis}{Hypothesis}

% Begin document
% --------------

\begin{document}


\begin{titlepage}
    \begin{center}
    \line(1,0){300} \\ 
    [0.25in]
    \huge{\bfseries Designing with the Mind in Mind} \\
    [2mm]
    \line(1,0){200} \\
    [1.5cm] 
    \textsc{\Large Chapters 10, 11 and 12} \\
    [0.75cm]
    \textsc{\Large Reading Summary} \\
    [12cm]
    \end{center}
    
    \begin{flushright}
    \textsc{\Large{Gurpreet Singh \\}\normalsize\emph{\ February 05, 2018 \\}\normalsize\emph{CS4474 \\} }
    
    \end{flushright}

\end{titlepage}

\newpage
\shipout\null

\hypertarget{chapter-10}{%
\section{Chapter 10}\label{chapter-10}}

\hypertarget{learning-from-experience}{%
\subsection{Learning From Experience}\label{learning-from-experience}}

\hypertarget{three-brains}{%
\subsubsection{Three Brains}\label{three-brains}}

The old brain categorizes everything into edible, dangerous, or sexy and
regulates the body's automatic functions. The mid brain controls
emotions and reactions to events. The new brain controls conscious
activity including planning.

When we run into a situation, all three brains decide how we react and
our actions. The old brain and mid brain react faster than the new
brain.

\hypertarget{learning-from-experience-is-easy}{%
\subsubsection{Learning from Experience is
Easy}\label{learning-from-experience-is-easy}}

People learn from their daily experiences without knowing that they are
learning. Some problems with learning from experience are listed
next.Complex situations with lots of variables are hard to learn from
because they do not occur as often as other experiences.

Real life experiences are more valuable to the brain then ones you read
or hear about. If something has happened to you or your family member,
you will remember it better.

You may not learn the correct lesson from an experience if you made the
wrong decision at the time of the event. You are not always able to
remember the correct decision if the event occurred again.

People often overgeneralize. If you have only seen dangerous animals
that are black, you will think all dangerous animals are black. This can
be seen as a con or a pro because overgeneralization allows humans to
make assumptions about things they havent seen before which could save
their life and help evolution.

\hypertarget{performing-learned-actions-is-easy}{%
\subsubsection{Performing learned actions is
easy}\label{performing-learned-actions-is-easy}}

When we do something many times we can do it without thinking. Real
world tasks have a mixture of automatic and controlled components
because without the mix we wouldn't be able to process everything at
once. We need automatic activities to assist us.

\hypertarget{problem-solving-and-calculation-are-hard}{%
\subsubsection{Problem Solving and Calculation are
Hard}\label{problem-solving-and-calculation-are-hard}}

New problems are hard to solve. Having a large new brain helps us solve
these problems. Problem solving requires focus and is slow. Executes
slowly and serially. Our brains are not optimized for calculation
because numbers are so new. We need external memory aids to solve
complex problems because we use them like extra working memory.

\hypertarget{implications-for-ui-design}{%
\subsubsection{Implications for UI
design}\label{implications-for-ui-design}}

\begin{itemize}
\item
  Indicate system status and how far user is to goal
\item
  Guide users toward the goal
\item
  Tell users instructions and make them exact
\item
  Don't make users diagnose system problems
\item
  Minimize number of settings
\item
  Minimize calculation (Use graphs)
\item
  Make system similar to something user is used to
\end{itemize}

\hypertarget{chapter-11}{%
\section{Chapter 11}\label{chapter-11}}

\hypertarget{many-factors-assist-learning}{%
\subsection{Many Factors Assist
Learning}\label{many-factors-assist-learning}}

\hypertarget{learn-faster-when-task-is-consistent}{%
\subsubsection{Learn faster when task is
consistent}\label{learn-faster-when-task-is-consistent}}

The gap between what the user wants and what a tool provides is called
the ``gulf of execution''. The smaller the gulf of execution the less
the users need to think about the tool. Design the tool to provide
exactly what the user wants to do

\begin{itemize}
\item
  Perform a task analysis
\item
  Design task focused conceptual model
\item
  Design UI based on conceptual model
\end{itemize}

A task analysis answers questions about what the user wants to do with
the software and which tasks are most important/frequent.

\hypertarget{object-action-analysis}{%
\subsubsection{Object / Action Analysis}\label{object-action-analysis}}

Specifies all the conceptual objects that an application will expose to
the user. Basically a UML diagram for UI objects.

Aim to simplify this with the least amount of concepts and actions the
user has to remember.

Interactive systems should aim to be consistent. More consistent they
are the faster the user can begin using automatic actions inside of the
software

You can make a matrix with Objects on Y and Actions on X and find out
which interactions are being done the most to optimize the usage
patterns

\hypertarget{learn-faster-when-vocab-is-task-focused}{%
\subsubsection{Learn faster when vocab is
task-focused}\label{learn-faster-when-vocab-is-task-focused}}

Keeping terminology related to the task the user is trying to finish
minimizes the time a user needs to learn how to use the software. Use
easier words that non CS people can understand. Public instead of DB and
private instead of local. Use words that are seen by many people often.
Use similar words around the whole software so the user has to learn
less.

\hypertarget{learn-faster-when-risk-is-low}{%
\subsubsection{Learn faster when Risk is
Low}\label{learn-faster-when-risk-is-low}}

When there is not much to lose the user will be more inclined to click
around and explore the software. If the user is scared of messing
something up they wont click on anything they arnt used to seeing the
result of. Make easily approachable UIs

\hypertarget{chapter-12}{%
\section{Chapter 12}\label{chapter-12}}

\hypertarget{time-requirements-exist}{%
\subsection{Time Requirements Exist}\label{time-requirements-exist}}

\hypertarget{responsivness-defined}{%
\subsubsection{Responsivness defined}\label{responsivness-defined}}

Responsiveness is how quickly a system responds to the user's
interactions. A system can be responsive even with poor performance. You
should use callbacks to let the UI continue operating and let the user
know that something will finish. When a system isnt responsive it cant
meet the time deadlines of the human.

\hypertarget{time-constraints-for-the-human-brain}{%
\subsubsection{Time constraints for the human
brain}\label{time-constraints-for-the-human-brain}}

1 milisecond is the shortest amount of silence you can detect. 5
miliseconds is the shortest amount of time you can see a visual change
to be effected by it 80 miliseconds is how long it takes you to flinch
to something. Then it takes 100 miliseconds to fully process something
you have seen. Your system has to react to an action within 140
miliseconds (1.4 seconds) for the user to understand that their action
had an effect on your system. Your attention takes 500 miliseconds to
reset from one item to the next. It takes 700 miliseconds to do a motor
action after observing something visual. Any gap in conversation longer
than 1 second is awkward. A sub-mental task can take up to a maximum of
10 seconds.

Interactive systems need to keep the above constraints in mind when
engineering.

\hypertarget{hci-implications}{%
\subsubsection{HCI implications}\label{hci-implications}}

The guidelines for this chapter are:

\begin{itemize}
\tightlist
\item
  React to user interactions instantly
\item
  Indicate background processing
\item
  Use callbacks and dont hold UI in focus when processing
\item
  Animate smoothly
\item
  Allow users to cancel processing
\item
  provide a ETA for processing
\end{itemize}

Additional implications include - use busy indicators (spinning circles
and loading bars) - Know when it is allowed to use a delay and when it
will bother the user (sometimes its acceptable if they are doing
something hard) - Display important data first and avoid delays by
prompting for extra information instead of preprocessing everything for
viewing - When the user is not doing anything directly, you can process
information in the background and be ready for the next actions the user
may take - Monitor how long your application has taken and if it falls
into an acceptable time duration

\end{document}

